\documentclass[a4paper,fleqn]{cas-dc}
% If the frontmatter runs over more than one page, use the longmktitle option.
%\usepackage[numbers]{natbib}
%\usepackage[authoryear]{natbib}
\usepackage[authoryear,longnamesfirst]{natbib}
\usepackage{float}
\usepackage{lipsum}

%%%Author macros
% Define a command \tsc for setting acronyms in small caps. 
% Usage: \tsc{<acronym>} will define a command for the acronym that outputs it in small caps.
\def\tsc#1{\csdef{#1}{\textsc{\lowercase{#1}}\xspace}}
\tsc{WGM}
\tsc{QE}
%%%

% Uncomment and use as if needed
%\newtheorem{theorem}{Theorem}
%\newtheorem{lemma}[theorem]{Lemma}
%\newdefinition{rmk}{Remark}
%\newproof{pf}{Proof}
%\newproof{pot}{Proof of Theorem \ref{thm}}

\begin{document}
% Adjusts PDF bookmarks, maximizes float pages usage, and minimizes text space for float placement in LaTeX.
\let\WriteBookmarks\relax
\def\floatpagepagefraction{1}
\def\textpagefraction{.001}

% Short title, short author
\shorttitle{<short title of the paper for running head>}
\shortauthors{<short author list for running head>}  

% Main title of the paper
\title [mode = title]{<Main title of the paper>}  

% Title footnote mark
\tnotemark[1,2]
\tnotetext[1]{This document is the results of the research project funded by the National Science Foundation.}
\tnotetext[2]{The second title footnote which is a longer text 
   matter to fill through the whole text width and overflow 
   into another line in the footnotes area of the first page.}

% First author
\author[1,3]{Author Name1}[type=editor,
                          auid=000,
                          bioid=1,
                          prefix=Sir,
                          role=Researcher,
                          orcid=0000-0001-0000-0000]
% Corresponding author indication, 1 for * and 2 for ** next to the author name
\cormark[1]
% Footnote of the first author, 1 for 1, 2 for 2 next to the author name
\fnmark[1]
% Email id of the first author
\ead{author1@example.in}
% URL of the first author
\ead[url]{https://www.baidu.com}
% Credit authorship, https://www.elsevier.com/researcher/author/policies-and-guidelines/credit-author-statement
\credit{Conceptualization of this study, Methodology, Software}
% Address/affiliation
\affiliation[1]{organization={State Key Laboratory of Ocean Engineering, Shanghai Jiao Tong University (SJTU)},
            % addressline={}, 
            % city={Shanghai},
%          citysep={}, % Uncomment if no comma needed between city and postcode
            postcode={200240}, 
            state={Shanghai},
            country={China}}

% Second author
\author[2,3]{Author Name2}[style=chinese]

% Third author
\author[2]{Author Name3}[
   role=Co-ordinator,
   suffix=Jr,
   ]
% Footnote of the second author
\fnmark[2]
% Email id of the second author
\ead{author2@example.com}
% URL of the second author
\ead[url]{wwww.google.com}
% Credit authorship
\credit{Data curation, Writing - Original draft preparation}
% Address/affiliation
\affiliation[2]{organization={SJTU-Sanya Yazhou Bay Institute of Deepsea Science and Technology},
            % addressline={}, 
            city={Sanya},
%          citysep={}, % Uncomment if no comma needed between city and postcode
            postcode={572024}, 
            state={Hainan},
            country={China}}

% Fourth author
\author[1,3]{Author Name4}
\cormark[2]
\fnmark[1,3]
\ead{author4@example.com}
\ead[url]{www.google.com}
\affiliation[3]{organization={Department of Mechanical Engineering, University of California},
            % addressline={}, 
            city={Berkeley},
%          citysep={}, % Uncomment if no comma needed between city and postcode
            postcode={94720}, 
            state={California},
            country={USA}}
        
% Corresponding author text
\cortext[1]{Corresponding author}
\cortext[2]{Principal corresponding author}

% Footnote text
\fntext[1]{This is the first author footnote, but is common 
   to third author as well.}
\fntext[2]{Another author footnote, this is a very long 
  footnote and it should be a really long footnote. But this 
  footnote is not yet sufficiently long enough to make two 
  lines of footnote text.}

% For a title note without a number/mark
\nonumnote{This note has no numbers in front.}

% Here goes the abstract
\begin{abstract}
  This template helps you to create a properly formatted \LaTeX\ manuscript. This template helps you to create a properly formatted \LaTeX\ manuscript. This template helps you to create a properly formatted \LaTeX\ manuscript. This template helps you to create a properly formatted \LaTeX\ manuscript. This template helps you to create a properly formatted \LaTeX\ manuscript.
\end{abstract}

% Keywords
% Each keyword is seperated by \sep
\begin{keywords}
  quadrupole exciton \sep polariton \sep WGM \sep BEC
\end{keywords}

% Use if graphical abstract is present
%\begin{graphicalabstract}
%\includegraphics{}
%\end{graphicalabstract}

% Research highlights
\begin{highlights}
\item This template helps you to create a properly formatted \LaTeX\ manuscript.
\item  This template helps you to create a properly formatted \LaTeX\ manuscript.
\item  This template helps you to create a properly formatted \LaTeX\ manuscript.
\end{highlights}

\maketitle

% Main text
\section{Introduction}

% Numbered list
% Use the style of numbering in square brackets.
% If nothing is used, default style will be taken.
\begin{enumerate}[1.]
  \item The enumerate environment starts with an optional argument `1.' so that the item counter will be suffixed by a period as in the optional argument.
  \item If you provide a closing parenthesis to the number in the optional argument, the output will have closing parenthesis for all the item counters.
  \item You can use `(a)' for alphabetical counter and `(i)' for roman counter.
  \begin{enumerate}[a)]
    \item Another level of list with alphabetical counter.
    \item One more item before we start another.
    \begin{enumerate}[(i)]
      \item This item has roman numeral counter.
      \item Another one before we close the third level.
    \end{enumerate}
    \item Third item in second level.
  \end{enumerate}
  \item All list items conclude with this step.
\end{enumerate}

% Description list
\begin{description}
\item I don't know what it is.
\item Maybe later I will know.
\item So I keep it here.
\end{description}

\lipsum[2]\cite{NewmanGirvan2004}

\section{Figures and Tables}
\lipsum[3-4]

% Figure
\begin{figure*}[pos=H]
	\centering
		\includegraphics[width=0.6\textwidth]{example.jpg}
	  \caption{example for span figure}\label{fig1}
\end{figure*}

\begin{figure}[pos=H]
	\centering
		\includegraphics[width=0.3\textwidth]{example.jpg}
	  \caption{example for normal figure}\label{fig2}
\end{figure}

% table
\begin{table*}[width=.9\textwidth,cols=4,pos=H]
  \caption{example for span table}
  \begin{tabular*}{\tblwidth}{@{}LLLLLLL@{}}
   \toprule
    Col 1 & Col 2 & Col 3 & Col4 & Col5 & Col6 & Col7\\
   \midrule
    12345 & 12345 & 123 & 12345 & 123 & 12345 & 123 \\
    12345 & 12345 & 123 & 12345 & 123 & 12345 & 123 \\
    12345 & 12345 & 123 & 12345 & 123 & 12345 & 123 \\
    12345 & 12345 & 123 & 12345 & 123 & 12345 & 123 \\
    12345 & 12345 & 123 & 12345 & 123 & 12345 & 123 \\
    12345 & 12345 & 123 & 12345 & 123 & 12345 & 123 \\
    12345 & 12345 & 123 & 12345 & 123 & 12345 & 123 \\
   \bottomrule
  \end{tabular*}
\end{table*}

\begin{table}[pos=H]
  \caption{example for normal table}
  \begin{tabular*}{\tblwidth}{@{} LLLL@{} }
   \toprule
    Col 1 & Col 2\\
   \midrule
    12345 & 12345\\
    12345 & 12345\\
    12345 & 12345\\
    12345 & 12345\\
    12345 & 12345\\
    12345 & 12345\\
   \bottomrule
  \end{tabular*}
\end{table}

% Uncomment and use as the case may be
%\begin{theorem} 
%\end{theorem}

% Uncomment and use as the case may be
%\begin{lemma} 
%\end{lemma}

%% The Appendices part is started with the command \appendix;
%% appendix sections are then done as normal sections

\appendix
\section{More details}
% Generate a paragraph for appendix
\lipsum[1]\cite{Raghavanetal2007}

% To print the credit authorship contribution details
\printcredits

%% Loading bibliography style file
%\bibliographystyle{model1-num-names}
\bibliographystyle{cas-model2-names}

% Loading bibliography database
\bibliography{cas-refs}

% % Biography
\bio{example.jpg}
% % Here goes the biography details.
\endbio

% \bio{pic1}
% Here goes the biography details.
% \endbio

\end{document}

